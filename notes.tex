% chktex-file 44

\documentclass[]{article}
\usepackage[margin=1.5cm]{geometry}
\usepackage{latexsym}
\usepackage{amsmath}
\usepackage{color}
\usepackage{graphicx}
\usepackage{amssymb}
\usepackage{alltt}
\usepackage{enumitem}
\usepackage{siunitx}
\usepackage{physics}
\usepackage{float}
\usepackage{tcolorbox}
\usepackage{relsize}
\usepackage{mathtools}
\usepackage{hyperref}

\setlength{\parindent}{0pt}
\renewcommand{\arraystretch}{1.7}

%%%%%%%%%%%%%%%%%%%%%%%%%%%%%%%%%%%%%%%%%%%%%%%%%%%%%%%%%%%%%%%%%%%%%%%%%%%%%%%%%%%%%%%%%%%%%%%%%%%%
%--------------------------------------------------------------------------------------------------%
%%%%%%%%%%%%%%%%%%%%%%%%%%%%%%%%%%%%%%%%%%%%%%%%%%%%%%%%%%%%%%%%%%%%%%%%%%%%%%%%%%%%%%%%%%%%%%%%%%%%

\begin{document}

\begin{table}[h]
    \centering
    \begin{tabular}{|l||l|}
        \hline
        Complex number (standard form)
        & $a + bi$
        \\
        Complex number (polar form)
        & $r(\cos(\theta) + i\sin(\theta)) = re^{i\theta}$
        \\
        Adding complex numbers
        & $(a + bi) + (c + di) = (a + c) + (b + d)i$
        \\
        Multiplying complex numbers
        & $r_1(\cos(\theta_1) + i\sin(\theta_1)) \cdot r_2(\cos(\theta_2) + i\sin(\theta_2)) =
        r_1r_2(\cos(\theta_1\theta_2) + i\sin(\theta_1\theta_2))$ \\
        & \textit{multiply the lengths, add the angles}
        \\
        Complex exponent
        & $e^{a + bi} = e^{a}e^{bi}$ \\
        & \textit{length $= e^{a}$, angle $= b$}
        \\
        \hline
    \end{tabular}
\end{table}

\end{document}
